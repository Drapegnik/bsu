\documentclass[oneside, final, 11pt]{article}
\usepackage[utf8]{inputenc}
\usepackage[T2A]{fontenc}
\usepackage[russian]{babel}
\usepackage{natbib}
\usepackage{graphicx}
\usepackage{amsmath}
\usepackage{xcolor}
\usepackage{amssymb}
\usepackage{graphicx}
\graphicspath{ {images/} }
\usepackage{subcaption}

\usepackage[paper=a4paper, margin=2cm, bottom=2cm]{geometry}

\newcommand\tab[1][1cm]{\hspace*{#1}}

\begin{document}
\begin{titlepage}
\newgeometry{margin=1cm}

\centerline{\large \bf МИНИСТЕРСТВО ОБРАЗОВАНИЯ РЕСПУБЛИКИ БЕЛАРУСЬ}
\bigskip
\bigskip
\centerline{\large \bf БЕЛОРУССКИЙ ГОСУДАРСТВЕННЫЙ УНИВЕРСИТЕТ}
\bigskip
\bigskip
\centerline{\large \bf ФАКУЛЬТЕТ ПРИКЛАДНОЙ МАТЕМАТИКИ И ИНФОРМАТИКИ}
\vfill
\vfill
\vfill
\centerline{\large \bf МЕТОДЫ ЧИСЛЕННОГО АНАЛИЗА}
\bigskip
\bigskip
\vfill
\begin{centering}
  \large Лабараторная работа №2 \\ 
 ``Дифференциальные уравнения в частных производных'' \\
  студента 3 курса 1 группы \\
  \bf Пажитных Ивана Павловича \\
\end{centering}
\vfill
\vfill
\hfill
\begin{minipage}{0.25\textwidth}
  {\large{\bf Преподаватель} \\
{\it Бондарь Иван \\ Васильевич}}
\end{minipage}
\vfill
\vfill
\centerline{\large \bf Минск 2016}

\end{titlepage}

\restoregeometry

\section{Условие}
\begin{itemize}
    \item Построить явную разностную схему
    \item Аппроксимировать граничные условия со вторым порядком
    \item Провести исследование порядка точности и устойчивости построенной разностной схемы
    \item Выполнить программную реализацию построенной схемы:
        \begin{itemize}
            \item $N = 50, 100, 200$
            \item Шаг $h$ определить из условий устойчивости
            \item $T=[0,1]$
            \item Построить графики
            \item Проверить точность решения
            \item Вывести время работы программы
        \end{itemize}
\end{itemize}

\section{Вариант №5}

\begin{equation}
    \frac{\partial u}{\partial t}-25\frac{\partial^2 u}{\partial x^2} = f(t,x)
\end{equation}

\begin{equation}
    f(t,x)=\pi(-12\pi(t-2)^2\sin(\pi(t-2)x)-x\cos(\pi(t-2)x))
\end{equation}

\begin{equation}
    \begin{cases}
        u(t,0)=0 \\
        u(0,x)=sin(2\pi x)
    \end{cases}
\end{equation}

\begin{equation}
    \frac{\partial u(t,1)}{\partial x} = -\pi(t-2)\cos(\pi(t-2))
\end{equation}

\begin{center}
    \bf{Точное решение}: 
\end{center}

\begin{equation}
    u(t,x)=-\sin(\pi(t-2)x)
\end{equation}

\section{Решение}

\subsection{Явная разностная схема для (1):}
\begin{equation}
    \frac{u_i^{k+1}-u_i^k}{\tau}-25\frac{u_{i-1}^k-2u_i^k+u_{i+1}^k}{h^2} = f(t,x)
\end{equation}

\subsection{Аппроксимация граничного условия (4):}
\subsubsection{с порядком точности $O(h)$:}
\begin{equation}
    \frac{u(t,1)-u(t,1-h)}{h} = -\pi(t-2)\cos(\pi(t-2))
\end{equation}

\subsubsection{с порядком точности $O(h^2)$:}
\begin{equation*}
    \psi = lhs(4) - lhs(7)
\end{equation*}

\begin{equation}
\begin{split}
    \psi = \frac{\partial u(t,1)}{\partial x} - \frac{u(t,1)-u(t,1-h)}{h} = \frac{\partial u(t,1)}{\partial x} - 
    \\
    -\frac{1}{h}\left(u(t,1) - u(t,1) + h\frac{\partial u(t,1)}{\partial x} - \frac{h^2}{2}\frac{\partial^2 u(t,1)}{\partial x^2} - O(h^3) \right) = \frac{h^}{2}\frac{\partial^2 u(t,1)}{\partial x^2} + O(h^2)
\end{split}
\end{equation}

\subsection{Порядок точности и устойчивость:}
\subsubsection{порядок точности (6):}
\begin{equation*}
    \psi = lhs(1) - lhs(6)
\end{equation*}

\begin{equation}
\begin{split}
    \psi = \frac{\partial u_i^k}{\partial t}-25\frac{\partial^2 u_i^k}{\partial x^2} - \left(\frac{u_i^{k+1}-u_i^k}{\tau}-25\frac{u_{i-1}^k-2u_i^k+u_{i+1}^k}{h^2}\right) = 
    \\
    =\frac{\partial u_i^k}{\partial t}-25\frac{\partial^2 u_i^k}{\partial x^2} 
    - \frac{1}{\tau}\left(u_i^k + \tau\frac{\partial u_i^k}{\partial t} + \frac{\tau^2}{2}\frac{\partial^2 u_i^k}{\partial t^2} + O(\tau^3) - u_i^k \right) +
    \\
    + \frac{25}{h^2}\left(\left(u_i^k - h\frac{\partial u_i^k}{\partial x} + \frac{h^2}{2}\frac{\partial^2 u_i^k}{\partial x^2} + \frac{h^4}{4!}\frac{\partial^4 u_i^k}{\partial x^4} + O(h^5)\right) - 2u_i^k +
    \\
    + \left(u_i^k + h\frac{\partial u_i^k}{\partial x} + \frac{h^2}{2}\frac{\partial^2 u_i^k}{\partial x^2} + \frac{h^4}{4!}\frac{\partial^4 u_i^k}{\partial x^4} + O(h^5) \right) \right) = 
    \\
    = \frac{\tau}{2}\frac{\partial^2 u_i^k}{\partial t^2} + 50\frac{h^2}{4!}\frac{\partial^4 u_i^k}{\partial x^4} + O(\tau^2 + h^3)
\end{split}
\end{equation}
\newline
порядок точности (7): $\psi \sim O(h)$ \\
порядок точности (8): $\psi \sim O(h^2)$ \\
\textbf{Итого порядок точности (6) c (8)}: $O(\tau + h^2)$

\subsubsection{устойчивость (6) методом гармоник:}
\begin{equation}
    \frac{\partial u}{\partial t}-25\frac{\partial^2 u}{\partial x^2} = 0
\end{equation}

\begin{equation}
    \frac{u_i^{k+1}-u_i^k}{\tau}-25\frac{u_{i-1}^k-2u_i^k+u_{i+1}^k}{h^2} = 0
\end{equation}

\begin{equation}
    u_i^k \sim q^ke^{i\phi_ih}
\end{equation}

подставляя (12) в (11):

\begin{equation}
\begin{split}
    \frac{q^{k+1}e^{i\phi_i h}-q^k e^{i\phi_i h}}{\tau}-25\frac{q^k e^{(i-1)\phi_i h}-2q^k e^{i\phi_i h}+q^k e^{(i+1)\phi_i h}}{h^2} = \left[ /q^k e^{i\phi_i h} \right] =
    \\
    = \frac{q-1}{\tau} + \frac{25}{h^2}(e^{-i\phi h} - 2 + e^{i\phi h}) = 0
\end{split}
\end{equation}

\begin{equation}
 |q| = \left| 1 -\frac{25\tau}{h^2}(e^{-i\phi h} - 2 + e^{i\phi h}) \right| = \left| 1 -\frac{25\tau}{h^2}(2\cos{\phi h} - 2) \right|  < 1
\end{equation}

раскрывая модуль в (14):

\begin{equation}
    -\frac{1}{25} < \frac{\tau}{h^2}(\cos{\phi h} - 1) < 0
\end{equation}

\textbf{Итого (6) устойчива при}:
\begin{equation}
    \frac{\tau}{h^2} > -\frac{1}{25}
\end{equation}

\end{document}
